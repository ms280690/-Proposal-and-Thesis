\documentclass[11pt]{article}

\usepackage{wasysym}     
\usepackage{hyperref}

\setlength{\marginparwidth}{1.2in}
\let\oldmarginpar\marginpar
\renewcommand\marginpar[1]{\-\oldmarginpar[\raggedleft #1]%
{\raggedright #1}}    

\newenvironment{checklist}{%
  \begin{list}{}{}% whatever you want the list to be
  \let\olditem\item
  \renewcommand\item{\olditem -- \marginpar{$\Box$} }
  \newcommand\checkeditem{\olditem -- \marginpar{$\CheckedBox$} }
}{%
  \end{list}
}   

\begin{document}
\begin{titlepage}
   \vspace*{\stretch{1.0}}
   \begin{center}
      \huge\textbf{The Thesis To-Do-List}\\
      \Large\textit{Mehul C. Solanki} \\
      \date{\today} 
      \large\texttt{solanki@unbc.ca} \\
   \end{center}
   \vspace*{\stretch{2.0}}
\end{titlepage}

\tableofcontents
\clearpage

\section{\textsc{Prolog} Grammar}
\begin{checklist}
  \checkeditem   Play around with and understand the parser and syntax of \textbf{prolog-0.2.0.1} library. 
  
  \checkeditem  Add floating point support. 
  
  \item   Study  syntax and grammar from 
 \href{http://sicstus.sics.se/sicstus/docs/3.7.1/html/sicstus_45.html}{\textsc{Sicstus Prolog}}
 and incorporate it into the parser above.
 
 \item Incorporate those tricky \textsc{Prolog} terms with single quotes along with  support for ASCII codes
 
 \item Go through Dr. Casperson's code for syntax, parser, print and quasi quoting.
 
 \item Add arithmetic support during parsing that is, if an atom is of the form,
 \\* 3 + 4 + 5
 \\* after parsing should be 
 \\* 12.
 
 \item  Other \textsc{Prolog} syntax such as equality expressions,
 \\* =(X,Y). or
 \\* =(1,1). or
 \\* =('a','b').
 
 \item The parser from the library parses grammars (CFG, DCG) but I do not know how well but everything works well. Moreover, there is no 
 special syntax support for recognising clauses of the grammar form.  
  
 \item Do something about quasi quotation and learn more about template \textsc{Haskell}.
 
 \item Look at \textbf{hswip} and \textbf{Nanoprolog} for anything new.
 
 \item Try and implement a small working example like \textit{append} which was mentioned in one of the papers. 
\end{checklist}

\section{Unification}
\begin{checklist}
\item Look at unification in \textsc{prolog-0.2.0.1}.

\item Look at unification-fd.

\item Look at monad-unify.

 \item Read about \textsc{Curry} and how they do residuation and narrowing in their most natural form, because the evaluation works just 
 like \textsc{Haskell} when the variables are confined to the left that is when the matching has to be done just one way.
 
 \item Look at the \textbf{Hugs98} for implementing variable search strategies. What is this "Andorra Engine" ?

\item Look at an unofficial implementaion at TAKASHI'S Workplace.

\item A new library called \textbf{compdata} which is based on 
\href{http://www.diku.dk/~paba/pubs/files/bahr11wgp-paper.pdf}{Compositional} 
\href{http://www.diku.dk/~paba/pubs/files/bahr11wgp-paper.pdf}{Data Types} and unifies terms using the 
 \href{http://www.nsl.com/misc/papers/martelli-montanari.pdf}{"Martelli and Montanari" Unification algorithm}.

\item If nothing else happens, try and implement the algorithm from above.

\end{checklist}

\section{IO and other practical features}
\begin{checklist}
\item Cut, how the hell does it work ?

\item Fail, how the hell does it work ?

\item Assert, how the hell does it work ?

\item SetOf, BagOf, how the hell does it work ?

\item consult , how the hell does it work ?

\item IO operations like reading a file, how the hell does it work ?
\end{checklist}

\section{Publications}
\begin{description}
\item[$\bullet$] Will functional programming take over the world?
\\* This paper ha by far the most content mostly thoughts and speculations based on experiences in Haskell collected from programming, 
reading and other sources. I feel this will be the last in terms of completion.  

Content is been added every day but the pace is slow.

\item[$\bullet$] Prolog in Haskell : A Survey.
\\* Now this is the one which is expected to be finished very soon even though it does not have much content in it.

Content is been added every week but the pace is very slow.

\item[$\bullet$] Prolog in Haskell : Reloaded.
\\* Probably after the work is near completed.

Content is been added once every ......  and the pace is very very slow.

\end{description}

\end{document}